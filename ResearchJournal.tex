\documentclass{scrartcl}

\usepackage[hidelinks]{hyperref}
\usepackage[none]{hyphenat}
\usepackage{setspace}
\doublespace
\usepackage{amsmath}

\usepackage{epigraph}
\usepackage{url}
\usepackage{graphicx}

\title{Concerning Go To Considered Harmful}
\subtitle{COMP210 - Research Journal}
\date{2018-10-10}
\author{1604281}

\begin{document}

\maketitle
\pagenumbering{arabic}


\section{Introduction}
Human computer interaction is an integral part of most software, including video games, as the users' successful interaction with the system is a key goal. Because of this, it is important to ensure that systems are intuitive and easy to use, allowing new users to interact without problems. Usability testing is therefore a vital part of software development to ensure the system usability is sufficient. 

This journal will examine several methods of usability testing, discussing their benefits and costs, as well as comparing their appropriateness to be used to research the usability of the Demiurge video game combat system.

\section{Game}
Demiurge is a third person bullet-hell game, in which the player must dodge projectiles or reflect them with their shield, while simultaneously attempting to aim their own projectiles to eliminate enemies. The game is a non-commercial prototype produced by a team of students. As the game is centred around the combat mechanics, the developers want to analyse the usability of the combat system, specifically how intuitive it is for the player and if the controls cause any confusion.


\section{Methods}

\subsection{Think-Aloud}
Think aloud testing is a basic but powerful form of usability testing where the users interact with the system while narrating their thoughts out loud for the observer to record. \cite{usabilityEngineering} This gives deep insight into the users' personal way of interacting with the system and quickly highlights areas of confusion or frustration.

There are two main methods of carrying out a Think-Aloud test, concurrent and retrospective. \cite{exploringThinkAloud} In a concurrent think-aloud, the user relays their thoughts in real time as they use the system. This has the advantage of being spontaneous and linked directly to the users' actions, however users may find it strange to perform in this manner. A retrospective has the user relate their thought process after using the system, possibly alongside a recording of them using the system. Although this may be a more comfortable method for the user, there is a chance that they will forget their exact thoughts, or spend more time thinking about what to say, thereby causing the test to be less genuine. \cite{eyeAndThink} Despite this, a retrospective think-aloud may allow users to more clearly relate complex problems they encountered with the interface, and explain their thought process in more depth . \cite{concurrentVRetro}

This method is also cheap and easy to perform, requiring only an observer, the users and the system, which can make it a good starting point to eliminate problems before considering more complex tests. This could be especially useful in the case of Demiurge, as the small team would be able to perform the tests without needing extensive resources. It would also give information on what parts of the combat the players were most interested in, which could help in balancing other aspects of the combat system, such as pacing or visual feedback.

However, without training or extensive research it can be difficult to analyse the results of a Think-Aloud - especially without a second method such as eye testing to compare with - as user thoughts can be vague and uninformative. This would necessitate an experienced usability tester to perform the tests, or to provide assistance or training to the team. \cite{assessingThinkAloud}


\subsection{Eye-tracking}
For more technical data about the system, eye-tracking tests can be conducted to discover how the user interacts with the interface at a deeper level. This method can help observers to understand the data from a think-aloud or similar test, by showing observers what the user was looking at while using the system. It is difficult for people to explain their thought process, as they can process thoughts far faster than they can articulate them. Monitoring eye movements gives an insight into the thought process, as people tend to be thinking about what they are looking at. \cite{eyeTracking} This also helps to prevent the observer from asking unnecessary questions, as they do not have to prompt the user for their thoughts on where they are looking. \cite{eyeAndThink}

Eye-tracking data can also be analysed separately from other usability tests, simply to provide insight into what parts of an interface drew the users attention the most. This can be used to test the effectiveness of specific system elements, such as adverts. 

As Demiurge is a fast-paced combat game, the ability to track player eye movements could help the developers understand how players gather information from the game, which could be used to inform UI design (to ensure it does not cover important areas of the screen) and combat elements, such as the range from which projectiles can be shot at the player. However, eye-tracking tests require specialised equipment, which is both expensive and needs training to use correctly. As the Demiurge team is a small group of students in a non-commercial environment, high-cost equipment would not be a practical method of testing despite its benefits.


\subsection{Surveys}
Surveys are a versatile method of data collection that can be tailored to suit any part of the system being tested. Of the largest benefits of surveys is their potential for collecting large amounts of data. As they do not require an observer or specialised equipment, they can be used to assess large groups of users, both in a testing environment, but also as continual testing after release, as surveys can be completed remotely by users. \cite{userExperience} Larger data sets provide greater opportunities to find meaningful trends and patterns that can be used to identify issues that effect a broad spectrum of users.




\section{Conclusion}



\bibliography{references}
\bibliographystyle{ieeetran}

\end{document}
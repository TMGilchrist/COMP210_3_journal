\documentclass{scrartcl}

\usepackage[hidelinks]{hyperref}
\usepackage[none]{hyphenat}
\usepackage{setspace}
\doublespace
\usepackage{amsmath}

\usepackage{epigraph}
\usepackage{url}
\usepackage{graphicx}

\title{Concerning Go To Considered Harmful}
\subtitle{COMP210 - Research Journal}
\date{2018-10-10}
\author{1604281}

\begin{document}

\maketitle
\pagenumbering{arabic}


\section{Introduction}
Human computer interaction is an integral part of most software, including video games, as the users' successful interaction with the system is a key goal. Because of this, it is important to ensure that systems are intuitive and easy to use, allowing new users to interact without problems. Usability testing is therefore a vital part of software development to ensure the system usability is sufficient. 

This journal will examine several methods of usability testing, discussing their benefits and costs, as well as comparing their appropriateness to be used to research the usability of the Demiurge video game combat system.

\section{Game}
Demiurge is a third person bullet-hell game, in which the player must dodge projectiles or reflect them with their shield, while simultaneously attempting to aim their own projectiles to eliminate enemies.


\section{Methods}

\subsection{Think Aloud}
Think aloud testing is a basic but powerful form of usability testing where the users interact with the system while narrating their thoughts out loud for the observer to record. This gives deep insight into the users' personal way of interacting with the system and quickly highlights areas of confusion or frustration.

This method is also cheap and easy to perform, requiring only an observer, the users and the system, which can make it a good starting point to eliminate problems before considering more complex tests.

\textsl{•}


\subsection{Eyetracking}
For more technical data about the system, eyetracking tests can be conducted to discover how the user interacts with the interface at a deeper level. This method can help observers to understand the data from a think aloud or similar test, by showing observers what the user was looking at while using the system. It is difficult for people to explain their thought process, as they can process thoughts far faster than they can articulate them. Monitoring eye movements gives an insight into the thought process, as people tend to be thinking about what they are looking at. /cite[eyeTracking]


\subsection{Surveys}



\section{Conclusion}



\bibliography{references}
\bibliographystyle{ieeetran}

\end{document}